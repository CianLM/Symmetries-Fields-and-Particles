\lecture{1}{11/10/2024}{Introduction}

\section{Introduction}

Symmetries are hidden throughout undergraduate physics. Lagrangian mechanics relies on the principle of least action, where the action $S$ is given by
\begin{align}
    S = \int_{t_1}^{t_2} \dd{t} L \left( q\left( t \right), \dot{q}\left( t \right) ; t  \right) 
.\end{align}

Classical trajectories minimise $S$ which gives us the Euler Lagrange equation,
\begin{align}
    \pdv{L}{q} - \dv{t} \left( \pdv{L}{\dot{q}} \right) = 0
.\end{align}

\begin{theorem}[ (Noether's Theorem)]
    Invariance of $L$ under some transformation implies an associated conserved quantity.
\end{theorem}

\begin{example}
    Take a particle in a 3-dimensional potential which has Lagrangian
    \begin{align}
        L = \frac{1}{2}m \left( \dot{x}^2 + \dot{y}^2 + \dot{z}^2 \right) - U \left( x,y,z \right) 
    .\end{align}

    There are a few notable symmetries here
    \begin{enumerate}
        \item $L$ is independent of time $t$, i.e. under $t \mapsto t + \delta t$.
            \begin{claim}
                The Hamiltonian $H = T + U$ is conserved.
            \end{claim}

            In general $H\left( x_{i}, p_{i} \right) $ is a function of $x_{i} = (x,y,z)$ and the conjugate momenta $p_{i} = \pdv{L}{\dot{x}_{i}} = m \dot{x}_{i}$ and is written in terms of the Lagrangian through Legendre transform as
            \begin{align}
                H \left( x_{i}, p_{i}; t \right) = \sum_{i}^{} \dot{x}_{i} \pdv{L}{\dot{x}_{i}} - L
            .\end{align}
            Therefore, if $L$ does not depend on time one has
            \begin{align}
                \dv{H}{t} = 0 - \pdv{L}{t}  = 0
            ,\end{align}
            where we have used the Euler Lagrange equations to make the first term vanish.

        \item If $L$ is invariant under $x \mapsto x + \delta x$,
            \begin{align}
                \pdv{L}{x} = 0 \overset{\text{EL}}{\implies} \pdv{L}{\dot{x}} = p_x \text{~is constant}
            .\end{align}
        \item If $L$ is invariant under rotations about the $z$ axis then the $z$-component of angular momentum $L_z = x p_y - y p_x$ is constant. 

            Similarly, in cylindrical coordinates $x = \rho \cos \theta$, $y = \rho \sin \theta$ and the Lagrangian becomes
            \begin{align}
                L = \frac{1}{2} \left( m \dot{\rho}^2 + \rho^2 \dot{\theta} + \dot{z}^2 \right) - U \left( \rho, z \right) 
            .\end{align}

            Therefore, $\pdv{L}{\theta} = 0 \implies \pdv{L}{\dot{\theta}} = m \rho^2 \dot{\theta} = x p_y - y p_x = \text{constant}$.
    \end{enumerate}
\end{example}


\subsection{Symmetry in Quantum Mechanics}

Given a system whose states are elements of a Hilbert space $\mathcal{H}$. Here, symmetry implies there exists some invertible operator $U : \mathcal{H} \to \mathcal{H}$ which preserves inner products, up to an overall phase $e^{i\phi}$ (e.g. expectation values, transition amplitudes).

\begin{definition}
    Let $\ket{\Phi}, \ket{\Psi}$ be any normalised vectors in $\mathcal{H}$. Denote $\ket{U \Psi} = U \ket{\Psi}$. $U$ is a \textbf{symmetry transformation operator} if
    \begin{align}
        \left| \bra{U \Phi} \ket{U \Psi} \right|  = \left| \bra{\Phi} \ket{\Psi} \right| 
    .\end{align}
\end{definition}

\begin{proposition}[(Wigner's theorem)]
    Symmetry transformation operators are either
    \begin{enumerate}[label=\alph*)]
        \item linear and unitary, or
        \item anti-linear and anti-unitary, meaning for $\alpha, \beta \in \C$,
            \begin{align}
                U \left( \alpha \ket{\Psi} + \beta \ket{\Phi} \right) = \alpha^{*} U \ket{\Psi} + \beta^{*} \ket{\Phi}
            ,\end{align}
            and
            \begin{align}
                \bra{U \Phi} \ket{U \Psi} = \bra{\Phi} \ket{\Psi}^{*}
            ,\end{align}
            respectively. 

            Most symmetries fall into the former category, but a notable exception is time-reversal symmetry, falling into the latter.
    \end{enumerate}
    
\end{proposition}

Suppose we have a system with time independent Hamiltonian $H$. We can write down the time evolution of operators in the Schrödinger picture (where the states depend on time and the operators are static) as
\begin{align}
    \ket{\Psi \left( t \right)} = e^{-i H t} \ket{\Psi \left( 0 \right)}
.\end{align}

Let's look at applying a symmetry operator $U$ in each of the cases above.

\begin{enumerate}[label=\alph*)]
    \item
        \begin{align}
            \bra{U \Phi \left( t \right) } \ket{U \Psi \left( t \right) } &= \bra{\Psi \left( t \right) } \ket{\Phi \left( t \right) } \\
            &= \bra{\Phi} e^{-i H t} \ket{\Phi \left( 0 \right) } (*)
        .\end{align}
        We should find the same result by transforming $\ket{\Psi \left( 0 \right) }$ before the evolution
        \begin{align}
            \ket{U \Psi \left( t \right) } = e^{-i H t} \ket{U \Psi \left( 0 \right) }
        ,\end{align}
        which implies
        \begin{align}
            \bra{U \Phi} \ket{U \Psi \left( t \right) } &= \bra{U \Phi} e^{-i H t} \ket{U \Psi \left( 0 \right) } \\
            &= \bra{ \Phi} U^{\dag} e^{-i H t} U \ket{\psi \left( 0 \right) } 
        .\end{align}

        By comparing this to $(*)$ we find that
        \begin{align}
            U^{\dag} e^{-i H t} U = e^{-i H t}
        .\end{align}
        Therefore $U$ commutes with the Hamiltonian, $\left[ U, H \right] = 0$.
\end{enumerate}

\begin{examples}~
    \begin{enumerate}[label=\arabic*)]
        \item If $H$ commutes with $p$, $H$ cannot depend on $x$ as $\left[ x_{i}, p_{j} \right] = i \delta_{ij} \neq 0$. Therefore $H$ is invariant under translations $x \to x + a$. One can construct a unitary operator that generates translations with $U = \exp \left( i \vb{p} \cdot \vb{a} \right) $.
        \item If $H$ is rotationally symmetric the angular momentum operator commutes with $H$.
    \end{enumerate}
\end{examples}

\section{Lie Groups and algebras}

\subsection{Lie Groups}

\begin{definition}
    A \textbf{group} is a set $G$ together with a binary operation $\circ$ such that the following properties hold
    \begin{enumerate}[label=\roman*)]
        \item Closure: $g_2 \circ g_1 \in G$, $\forall g_1, g_2 \in G$,
        \item Associativity: $g_3 \circ \left( g_2 \circ g_1 \right) = \left( g_3 \circ g_2 \right)  \circ g_1$, $\forall g_1, g_2, g_3 \in G$,
        \item Identity: $\exists e \in G$ such that $g \circ e = e \circ g = g$, $\forall g \in G$,
        \item Inverse: $\forall g \in G$, $\exists g^{-1} \in G$ such that $g \circ g^{-1} = e = g^{-1} \circ g$.
    \end{enumerate}
\end{definition}

The identity $e$ and inverse of $g$ are unique.

\begin{definition}
    A group $\left( G, \circ \right) $ is \textbf{commutative} (\textbf{abelian}) if
    \begin{align}
        g_1 \circ g_2 = g_2 \circ g_1
    ,\end{align}
    $\forall g_1, g_2 \in G$. Otherwise $G$ is \textbf{non-commutative} (\textbf{non-abelian}).
\end{definition}





