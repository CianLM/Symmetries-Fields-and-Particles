\lecture{4}{19/10/2024}{The Exponential Map}

\section{The Exponential Map}



Observe that
\begin{align}
    T_e \left( \mathcal{O}\left( n \right)  \right) = T_e \left( SO \left( n \right)  \right) 
,\end{align}
as $\det I = 1$, so all curves passing through $I$ have $\det M = 1$.

\subsection{Unitary Groups}



Let $M\left( t \right) $ be a curve in $SU \left( n \right) $ with $ M \left( 0 \right) = I$. For small $t$, write $M \left( t \right) = I + t X + \mathcal{O}\left( t^2 \right) $, where $X = \dv{M}{t} \bigg|_{t=0}$. 

Unitarity of $M$ provides that for all $t$,
\begin{align}
    I &= M^{\dag} M \\
    U &= I + t \left( X + X^{\dag} \right) + \mathcal{O}\left( t^2 \right) 
,\end{align}
which implies $X^{\dag} = -X$, namely, elements of the tangent space are \textit{anti-Hermitian}.

\begin{claim}
    $\tr X = 0$ for $X \in L \left( SU \left( n \right)  \right)$ or $M \in SU \left( n \right) $
\end{claim}

\begin{proof}
    Look at
    \begin{align}
        M\left( t \right) = \mqty( 1 + t X_{11} & t X_{12} & \cdots & t X_{1n} \\
        t X_{21} & 1 + t X_{22} & \cdots & tX_{2n} \\
        \vdots & \vdots & \ddots & \vdots \\
        t X_{n1} & t X_{n2} & \cdots & 1 + t X_{n n} )
    .\end{align}
    Notice that
    \begin{align}
        1 = \det M = 1 + \underbrace{t \tr X}_{0} + \mathcal{O} \left( t^2 \right) 
    ,\end{align}
    where the underbraced term (and higher order ones) must vanish.
\end{proof}

For $U \left( n \right) $, $X$ can have non-zero trace.

\subsection{Lie algebra of a matrix Lie group}

Consider two curves $g_1 \left( x \left( t \right)  \right)$ and $g_2 \left( x \left( t \right)  \right)$ through the identity $e$ of some Lie group $G$. We define
\begin{align}
    X_1 := \dot{g}_1 \bigg|_{t=0}, &&  X_2 := \dot{g}_2 \bigg|_{t=0}
.\end{align}

One can define a product
\begin{align}
    g_3 \left( z\left( t \right)  \right) = g_2\left( y\left( t \right)  \right) g_1 \left( x \left( t \right)  \right) \in G
,\end{align}
satisfying
\begin{align}
    \dot{g}_3 \bigg|_{t=0} &= \left( \dot{g}_2 g_1 + g_2 \dot{g}_1 \right) \bigg|_{t=0}  \\
    &= X_2 + X_1 \in T_e \left( G \right)
,\end{align}
another vector in the tangent space.

The Lie bracket arises from the \textit{group commutator}.

\begin{definition}
    The \textbf{group commutator} of $g_1, g_2 \in G$, is
    \begin{align}
        \left[ g_1, g_2 \right]_{G} := g^{-1}_1 g_2^{-1} g_1 g_2 := h \in G
    .\end{align}
\end{definition}

Returning to our two curves through the identity $e$, $g_i \left( t \right) $ for $i \in \{1,2\} $, we can expand
\begin{align}
    g_i \left( t \right) = e + t X_i + t^2 W_i + \mathcal{O}\left( t^3 \right) 
.\end{align}

We have that
\begin{align}
    g_1 \left( t \right) g_2 \left( t \right) = e + t \left( X_1 + X_2 \right) + t^2 \left( X_1 X_2 + W_1 + W_2 \right) + \mathcal{O}\left( t^3 \right) 
,\end{align}
and
\begin{align}
    g_2 \left( t \right) g_1 \left( t \right) = e + t \left( X_1 + X_2 \right) + t^2 \left( X_2 X_1 + W_1 + W_2 \right) + \mathcal{O}\left( t^3 \right) 
.\end{align}

If we then look at
\begin{align}
    h \left( t \right) = \left[ g_2 \left( t \right) g_1\left( t \right)  \right]^{-1} g_1\left( t \right) g_2\left( t \right) = e + t^2 \underbrace{\left( X_1 X_2 - X_2 X_1 \right)}_{\left[ X_1, X_2 \right] }  + \cdots
,\end{align}
and thus the group commutator induces the Lie bracket in the algebra. As $h \left( t \right) \in G$, the tangent to $h\left( t \right)$ at $e$ is $\left[ X_1, X_2 \right] \in L \left( G \right) $, and thus we have closure under the Lie bracket.

\begin{itemize}
    \item We write the tangent space to a matrix Lie group $G \overset{\text{subgroup}}{<} GL \left( n, \mathbb{F} \right) $ at a general element $p$ as $T_{p} \left( G \right) $. Let $g \left( t \right) $ be a curve in the manifold through $p$ with $g \left( t_0 \right) = p$, and thus
        \begin{align}
            g \left( t + \epsilon \right) = g\left( t_0 \right) + \dot{\epsilon}\left( t_0 \right) + \mathcal{O}\left( \epsilon^2 \right) 
        .\end{align}
        As both $g \left( t_0 \right), g\left( t_0 + \epsilon \right) \in G$, there exists $h_p \left( \epsilon \right) \in G $ such that
        \begin{align}
            g \left( t_0 + \epsilon \right) = g \left( t_0 \right) h_p \left( \epsilon \right) 
        ,\end{align}
        and as $\epsilon \to 0$, $h_p \left( \epsilon \right) \to e$. For small $\epsilon$,
        \begin{align}
            h_p \left( \epsilon \right) = e + \epsilon X_p + \mathcal{O}\left( \epsilon^2 \right) 
        ,\end{align}
        for some $X_p \in L \left( G \right) = T_e \left( G \right)  $. Neglecting $\mathcal{O}\left( \epsilon^2 \right) $,
        \begin{align}
            e + \epsilon X_p = h_p \left( \epsilon \right) &= g^{-1}\left( t_0 \right) g\left( t_0 + \epsilon \right)  \\
            &= g^{-1} \left( t_0 \right) \left[ g \left( t_0 \right) + \epsilon \dot{g}\left( t_0 \right)  \right]   \\
            &= e + \epsilon \underbrace{g^{-1}\left( t_0 \right) \dot{g}\left( t_0 \right)}_{X_p}
        .\end{align}

\end{itemize}

\begin{claim}
    Conversely, for any $X \in L \left( G \right) $, there exists a unique curve $g \left( t \right) $ with $g^{-1} \left( t \right) \dot{g}\left( t \right) = X$ and $g \left( 0 \right) = g_0$.
\end{claim}
\begin{proof}
    This is a consequence of existence and uniqueness of solutions of ODEs. The solution of this ODE is
    \begin{align}
        g \left( t \right) = g_0 \exp \left( tX \right) 
    ,\end{align}
    where
    \begin{align}
        \exp t X := \sum_{k=0}^{\infty} \frac{\left( t X \right)^{k}}{k!} 
    .\end{align}
\end{proof}

\subsection{One parameter subgroups}

Given an $X \in L \left( G \right) $ , the curve
\begin{align}
    g_X \left( t \right) = \exp t X
,\end{align}
forms an \textit{abelian} subgroup of $G$, \textit{generated} by $X$.

Notice that $g_X \left( t \right) $ is isomorphic to the group of real numbers under addition $\left( \R, + \right) $ if only $g_X \left( 0 \right) = e$. If there exist other $t_0 \neq 0$ such that $g_X\left( t_0 \right) = 0 $, then we have periodic structure and then $g_X \left( t \right) $ is isomorphic to the circle $S^{1}$.





