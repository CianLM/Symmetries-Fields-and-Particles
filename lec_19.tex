\lecture{19}{23/11/2024}{Geometry of Roots}

\begin{claim}
    The roots in $\Phi$ span $\mathfrak{h}^{*}$. Recall $\left| \Phi \right| > \dim h^{*}$.
\end{claim}
\begin{proof}
    Suppose that the roots do not span $\mathfrak{h}^{*}$. Then there exists some orthogonal subspace, i.e. $\exists \lambda \in \mathfrak{h}^{*}$ such that $\left( \lambda, \alpha \right) =  0$, $\forall \alpha \in \Phi$.

    The corresponding vector in $\mathfrak{h}$ is
    \begin{align}
        H_\lambda \equiv \lambda^{i} H_i \in \mathfrak{h}
    .\end{align}
    As $\mathfrak{h}$ is the Cartan subalgebra, $\left[ H_{\lambda}, H \right] = 0$, $\forall H \in \mathfrak{h}$. We also know that $\left[ H_\lambda, E_\alpha \right] = \lambda^{i} \left[ H_i , E_\alpha \right] = \lambda^{i} \alpha_i E_\alpha = \left( \lambda, \alpha \right) E_\alpha = 0$, $\forall \alpha \in \Phi$.

    So $H_\lambda$ commutes with all $x \in \mathfrak{g}$ and therefore $\text{span}_\C \{H_\lambda\} $ is a nontrivial abelian ideal which contradicts $\mathfrak{g}$ being semisimple. Thus the roots must span $\mathfrak{h}^{*}$.
\end{proof}

Choose a basis $\{ \alpha_{\left( i \right) } \in \Phi  \mid i = 1, \cdots, r\} $ where the parentheses indicate we are discussing a basis. We than write
\begin{align}
    \mathfrak{h}^{*} = \text{span}_\C \{\alpha_{\left( i \right) }\}
.\end{align}

We call these \textbf{simple roots}.

Let $\mathfrak{h}^{*}_\R \subset \mathfrak{h}^{*}$ be the real vector space
\begin{align}
    \mathfrak{h}^{*}_\R = \text{span}_\R \{\alpha_{\left( i \right) }\} 
,\end{align}
for $i = 1, \cdots, r$.

\begin{claim}
    $\mathfrak{h}^{*}_\R$ contains all roots.
\end{claim}

\begin{proof}
    Let $\beta \in \Phi$. We can find coefficients such that $\beta = \beta^{i}\alpha_{\left( i \right) }$.
    Taking an inner product, we see
    \begin{align}
        \left( \beta, \alpha_{\left( j \right) } \right) = \beta^{i} \left( \alpha_{\left( i \right) }, \alpha_{\left( j \right) } \right) 
    .\end{align}
    As we established these inner products are real, and the equations above are non-degenerate, we must have real coefficients $\beta^{i} \in \R$. Thus $\beta \in \mathfrak{h}_{\R}^{*}$.
\end{proof}

\begin{claim}
    $\forall \lambda \in \mathfrak{h}_\R^{*}$, $\left( \lambda, \lambda \right) \geq 0$, with $\left( \lambda, \lambda \right) = 0 \iff \lambda = 0$
\end{claim}

\begin{proof}
    Recall
    \begin{align}
        \left( \lambda, \lambda \right) = \frac{1}{\mathcal{N}} \sum_{\gamma \in \Phi}^{} \left( \lambda, \gamma \right) \left( \gamma, \lambda \right)  = \frac{1}{\mathcal{N}} \sum_{\gamma \in \Phi}^{} \left( \lambda, \gamma \right)^2 \geq 0
    .\end{align}
    Further, $\left( \lambda , \lambda\right) = 0 \iff \left( \lambda, \gamma \right) = 0$, $\forall \gamma \in \Phi$, including $\{\alpha_{\left( i \right) }\}$. If the coefficients are zero for the basis set then we must have $\lambda = 0$.
\end{proof}

\begin{definition}
    Define $\left| \alpha \right| = \sqrt{\left( \alpha , \alpha \right) } $ as the \textbf{length} or \textbf{norm} of a root $\alpha \in \Phi$.
\end{definition}

This also holds identically for any $\lambda \in \mathfrak{h}^{*}_\R$.

For any $\alpha, \beta \in \Phi$, there is an \textbf{angle} $\theta$ such that
\begin{align}
    \left( \alpha, \beta \right) = \left| \alpha \right| \left| \beta \right| \cos \theta
.\end{align}

Then the quantization condition which gave us $\frac{2 \left( \alpha, \beta \right) }{\left( \alpha , \alpha \right) } \in \Z$ gives us that
\begin{align}
    \frac{2 \left( \alpha, \beta \right) }{\left( \alpha , \alpha \right) } = 2 \frac{\left| \beta \right| }{\left| \alpha \right| } \cos \theta \in \Z && \frac{2 \left( \alpha, \beta \right) }{\left( \beta, \beta \right) } = 2 \frac{\left| \alpha \right| }{\left| \beta \right| } \cos \theta \in \Z
.\end{align}
Multiplying these together, we see
\begin{align}
    4 \cos^2 \theta \in \Z
,\end{align}
and thus as $0 \leq \cos^2 \theta \leq 1$, this must be $0$, $1$, $2$ $3$, or $4$. Therefore with $\cos \theta = \pm \sqrt{n}/2$, for $n \in Z$, the angle is restricted to
\begin{align}
    \left| \theta \right| = \left\{ 0, \frac{\pi}{6}, \frac{\pi}{4}, \frac{\pi}{3}, \frac{\pi}{2}, 2 \frac{\pi}{3} 3 \frac{\pi}{4}, 5 \frac{\pi}{6} , \pi \right\} 
.\end{align}

\subsection{Simple roots}

Recall that $\left| \Phi \right| > \dim \mathfrak{h}^{*}_\R = r$. We pick a hyperplane of dimension $r = 1$ to partition $\mathfrak{h}_\R^{*}$ in two. The hyperplane includes the origin and excludes any roots. 

The roots are divided into two sets of equal number. Call one set positive roots and the other negative such that
\begin{align}
    \Phi = \Phi_+ \cup \Phi_-
,\end{align}
where $\alpha \in \Phi_+ \iff -\alpha \in \Phi_-$ and $\alpha , \beta \in \Phi_+ \implies \alpha + \beta \in \Phi_+$.

\begin{definition}
    A \textbf{simple root} is a positive root which cannot be written as a sum of other positive roots. The set of simple roots is denoted by $\Phi_S$.
\end{definition}

We will see that $\Phi_S$ forms a basis for $\mathfrak{h}^{*}_\R$.

\begin{claim}
    If $\alpha , \beta \in \Phi_S$ then $\alpha - \beta \notin \Phi$.
\end{claim}
\begin{proof}
    We proceed by contradiction. Suppose $\alpha - \beta \in \Phi$. 
    \begin{enumerate}[label=\alph*)]
        \item If $\alpha - \beta \in \Phi_+$, then $\alpha = \left( \alpha - \beta \right) + \beta$ which implies $\alpha$ is not a simple root which is a contradiction.
        \item If $\alpha -\beta \in \Phi_-$, then $\beta - \alpha \in \Phi_+$ and thus $\beta = \left( \beta - \alpha \right) + \alpha$ and thus $\beta$ is not a simple root.
    \end{enumerate}
\end{proof}

Recall the $\alpha$-string through $\beta$
\begin{align}
    S_{\alpha, \beta} = \left\{ \beta + \rho \alpha  \mid \rho \in \Z, n_- \leq \rho \leq n_+ \right\} 
,\end{align}
with $n_+ + n_- = - \frac{2 \left( \alpha, \beta \right) }{\left( \alpha, \alpha \right) } \in Z$.

\begin{definition}
    The \textbf{length} of a root string is
    \begin{align}
        \ell_{\alpha, \beta} = n_{+} - n_- + 1
    .\end{align}
\end{definition}

For $\alpha, \beta \in \Phi_S$, the $\alpha$-root string through $\beta$ has length $\ell_{\alpha, \beta} = n_+ + 1 = -\frac{2 \left( \alpha, \beta \right) }{\left( \alpha, \alpha \right) } + 1$.

By the above claim, $n_- = 0$.

\begin{claim}
    $\left( \alpha, \beta \right) \leq 0$ for $\alpha , \beta \in \Phi_S$ and $\beta \neq \alpha$.
\end{claim}
\begin{proof}
    This follows from $\ell_{\alpha, \beta} \geq 1$ and $\left( \alpha, \alpha \right)  > 0$.
\end{proof}


\begin{claim}
    Any $\beta \in \Phi_+$ can be written as a linear combination of simple roots with integer coefficients.
\end{claim}

\begin{proof}
    If $\beta \in \Phi_S$, we are done. If $\beta \notin \Phi_S$, then $\beta = \beta_1 + \beta_2$. If $\beta_1$ and $\beta_2$ are simple roots, we are done. Otherwise one iterates and thus terminates as $\dim \mathfrak{h}^{*}_\R = r$ is finite.
\end{proof}

\begin{claim}
    All roots $\alpha \in \Phi$ can be written
    \begin{align}
        \alpha = \sum_{i}^{}  p_i \alpha_{\left( i \right) }
    ,\end{align}
    with $p_{i} \in \Z$ and $\alpha_{\left( i \right) } \in \Phi_{S}$.
\end{claim}

By the properties above, all $p_i$ have the same sign or are 0.

\begin{claim}
    The simple roots are linearly independent.
\end{claim}



