\lecture{12}{07/11/2024}{Lorentz Representations}

We can equivalently write these matrices as
\begin{align}
    M^{0j} = K_j && \epsilon_{ijk} J_k
,\end{align}
where
\begin{align}
    \tensor{\left( M^{\mu \nu} \right) }{^{\alpha}_{\beta}} = \eta^{\mu \alpha} \delta^{\nu}_{\beta} - \eta^{\nu \alpha} \delta^{\mu}_{\beta}
.\end{align}

Then $\Lambda \in SO \left( 1,3 \right)^{+}$ can be written
\begin{align}
    \Lambda = \exp \left( \frac{1}{2} \omega_{\mu \nu} M^{\mu \nu} \right) = \exp \left( \theta^{i} J_i + \psi^{i} K_i \right) 
,\end{align}
where $\theta^{i}, \psi^{i} \in \R$ and $\omega_{\mu \nu} = - \omega_{\nu \mu}$ as before.

We can simplify these brackets by complexifying such that
\begin{align}
    L_i \equiv \frac{1}{2} \left( J_i + i K_i \right) \in \mathfrak{sl}\left( 2,\C \right)_{\C} \\
    R_i \equiv \frac{1}{2} \left( J_i - i K_i \right) \in \mathfrak{sl}\left( 2,\C \right)_{\C}
.\end{align}

Then, we see
\begin{align}
    \left[ L_i, L_j \right] = \epsilon_{ijk} L_{k} && \left[ R_{i}, R_{j} \right] = \epsilon_{ijk} R_{k}
,\end{align}
where for all $i,j$
\begin{align}
    \left[ L_{i}, L_{j} \right] = 0
.\end{align}

Then given a generic linear combination of $\theta^{i} J_i + \psi^{i} K_i \in \mathfrak{sl}\left( 2, \C \right) $ where $\theta^{i}, \psi^{i} \in \R$, complexifying gives us $\alpha^{i} L_i + \beta^{i} R_i \in \textbf{sl}\left( 2, \C \right)_{\C}$ where $\alpha, \beta \in \C$.

As they commute, we see that $\alpha^{i}L_i$ alone are elements of $\mathfrak{su}\left( 2 \right)_{\C}$ and identically for $\beta^{i} R_i$. Thus we see that
\begin{align}
    \textbf{sl}\left( 2,\C \right)_{\C} &\cong \mathfrak{su}\left( 2 \right)_{\C} \oplus \mathfrak{su}\left( 2 \right)_{\C} \\
    &\cong \mathfrak{sl}\left( 2,\C \right) \oplus \mathfrak{sl}\left( 2,\C \right) 
.\end{align}

Given $\Lambda \in G = \{\exp \left( x^{i} L_i + \beta^{i} R_i \right)  \mid  x^{i}, \beta^{i} \in \C\} $, we have as $\left[ L_i, R_i \right] = 0$
\begin{align}
    \Lambda &= \exp \left( \alpha^{i}L_i \right) \exp \left( \beta^{i}R_i \right) \\
    &= U_L U_R =: \left( U_L, U_R \right)
.\end{align}

This forms a direct product group.

\begin{note}
    A \textbf{direct product group} $G = A \times B$ is formed from groups $A$ and $B$ given by
    \begin{align}
        G = \{\left( a,b \right)  \mid a \in A, b \in B \} 
    ,\end{align}
    with operation $g, g' \in G$ given by
    \begin{align}
        g g' = \left( a a', b b' \right) 
    .\end{align}
    $A$ and $B$ are normal subgroups of the direct product group.
\end{note}

We can see that this applies to $\Lambda_1 \Lambda_2 = \left( U_{1L} U_{2L}, U_{1R} U_{2R} \right)$. 

We can form a representation of a direct product group by taking representations of the two subgroups,
\begin{align}
    D^{G}\left( \left( a,b \right)  \right) = D^{A}\left( a \right) \otimes D^{B}\left( b \right) 
.\end{align}

Hence $A = \{\exp \left( \alpha^{i}L_i \right)  \mid \alpha^{i} \in \C\} $ and analogously for $B$,
\begin{align}
    D^{G}\left( e^{\alpha^{i}L_i} e^{\beta^{i} R_i} \right) = D^{A}\left( e^{\alpha^{i}L_i} \right) \otimes D^{B} \left( e^{\beta^{i} R_i} \right) 
.\end{align}

For the Lie algebra, we then have
\begin{align}
    d^{L\left( G \right) }\left( \alpha L + \beta R \right) = \alpha d^{L\left( A \right) } \left( L \right) \otimes I + \beta I \otimes d^{\left( B \right) }\left( R \right) 
,\end{align}
which are representations of $\mathfrak{sl}\left( 2,\C \right)_{\C}$. Specifically for
\begin{align}
    J_i = L_i + R_i && K_i = =i \left( L_i - R_i \right) 
.\end{align}

For the Lorentz algebra, it is more useful to label our representations with
\begin{align}
    d^{\left( j_1,j_2 \right) }\left( J_{i} \right) = d^{\left( j_1 \right) }\left( T_i \right) \otimes I + I \otimes d^{\left( j_2 \right) }\left( T_i \right) 
,\end{align}
where $\{T_i\} $ is a basis for $\mathfrak{su}\left( 2 \right)_{\C}$. Identically for the boosts, we see
\begin{align}
    d^{\left( j_1,j_2 \right) }\left( K_{i} \right) = =i \left(  d^{\left( j_1 \right) }\left( T_i \right) \otimes I - I \otimes d^{\left( j_2 \right) }\left( T_i \right)  \right) 
.\end{align}

$2j_1$ and $2j_2$ are the highest weights of some $\mathfrak{su}\left( 2 \right) $ irreducible representations.

\begin{examples}~
    \begin{enumerate}[label=\alph*)]
        \item $\left( 0,0 \right) $ is the trivial representation corresponding to scalars.
        \item $\left( \frac{1}{2},0 \right) $ is a spinor: The \textit{fundamental representation} of $\mathfrak{sl}\left( 2,\C \right) $. Also a \textbf{Weyl spinor} and is left-handed.
        \item $\left( 0,\frac{1}{2} \right) $ is also a (Weyl) spinor and is conjugate to the fundamental representation. It is right handed.
        \item $\left( \frac{1}{2},\frac{1}{2} \right) $ gives us a 4-vector representation. Under $SO \left( 3 \right) $ rotations, this irreducible representation is reducible such that $\vb{2} \otimes \vb{2} = \vb{1} \oplus \vb{3}$. This can be thought of as $\left( x^{0}, x^{i} \right) $ where $x^{0}$ is trivial under such rotations and $x^{i}$ obviously mixes. This is not the case for boosts as $\left( \frac{1}{2},\frac{1}{2} \right) $ is irreducible.
    \end{enumerate}
\end{examples}

\begin{note}
    A Dirac spinor is given by $\left( \frac{1}{2}, 0 \right) \oplus \left( -,\frac{1}{2} \right) $.
\end{note}

\subsection{Poincare group and algebra}

% Lorentz group 

The Poincare group is the Lorentz group (rotations and boosts) with spacetime translations as well.

This is an example of an isometry group of Minkowski space, i.e. one which preserves distances in some sense (Lorentz scalar product). We write it as $ISO \left( 1,3 \right) $.

This is a semi direct product group. Namely,
\begin{align}
    ISO \left( 1,3 \right) \cong O \left( 1,3 \right) \ltimes T^{1,3}
,\end{align}
where $T^{1,3} \cong \left( \R^{1,3}, + \right) $ is the spacetime translation group. $T^{1,3}$  is a normal subgroup, but $O\left( 1,3 \right) $ is not.

Let $G$ be a group with a normal subgroup $N \trianglelefteq G$ and another subgroup $H \leq G$, not necessarily normal. Let $\phi$ be the group homomorphism $\phi : H \to \text{Aut}\left( N \right) $. Namely, for $h \in H$, $n \in N$, $\exists \phi_h : N \to N$ such that
\begin{align}
    \phi_h \left( n \right)  = h n h^{-1} \in N
,\end{align}
since $N$ is normal. The group $G'$ consists of pairs $G' = \{\left( n,h \right)  \mid n \in N,~ h \in H\} = H \ltimes N$ with group operation
\begin{align}
    \left( n_2, h_2 \right) \left( n_1,h_1 \right) = \left( n_2, \phi_{h_2}\left( n_1 \right), h_2 h_1  \right) 
.\end{align}

We have $G' \cong G$ where $G$ is the semi-direct product group.

