\lecture{5}{22/10/2024}{Lie Groups from Lie Algebras}

\newcommand{\id}{\text{id}}

\subsection{Lie Groups from Lie Algebras}

\begin{definition}
    Given a Lie algebra $L\left( G \right)$ of a Lie group $G$, we can define the \textbf{exponential map}:
    \begin{align}
        \exp : L \left( G \right) \to G
    ,\end{align}
    which for matrix Lie groups, is
    \begin{align}
        X \mapsto \exp X = \sum_{k=0}^{\infty}  \frac{X^{k}}{k!}
    .\end{align}
\end{definition}

Locally, the map is bijective (one to one). For the proof see Hall Section 2.7.
Globally, the map is generally, not.

\begin{example}
    For example, $U \left( 1 \right) = \{e^{i \theta}  \mid  \theta \in [ 0,2\pi ) \} $ and we have
    \begin{align}
        L \left( U \left( 1 \right)  \right) = \{i x  \mid  x \in \R\}
    ,\end{align}
    where clearly $\exp \left( ix \right) $ is not one to one globally since $e^{2\pi ni} = 1$, $\forall n \in \Z$.
\end{example}

\begin{example}
    $G = O \left( n \right) $. Let $X \in L \left( O \left( n \right)  \right) \subset \text{Skew}_n\left( \R \right) $. Let $M = \exp t X$, and observe that as $X$ is antisymmetric, $M^{T} = \left[ \exp X \right]^{T} = \exp \left( -tX \right) $. Therefore,
    \begin{align}
        M M^{T} = I = M^{T} M
    ,\end{align}
    and thus we recover $M \in O \left( n \right) $.
\end{example}

\begin{note}
    $\tr X = 0$. Let $\lambda_1, \cdots, \lambda_n$ be eigenvalues of $X$, and observe that
    \begin{align}
        \det M &= \det \left( \exp t X \right)  \\
        &= \exp \left( \tr t X \right)  \\
        &= \exp \left( 0 \right)  \\
        &= 1
    .\end{align}
    and thus $M  \in SO \left( n \right)$. Thus elements of $O \left( n \right) $ with determinant $-1$ are not in the image of the exponential map.
\end{note}

Therefore, $O \left( n \right) $ is a disconnected manifold. One can think of $O\left( n \right) $ as two disconnected islands, one with $\det M = 1$ containing the identity called \textit{proper rotations}, and another containing elements with $\det M = -1$ called \textit{improper rotations} as they contain a reflection.

% fig

One can show that $A \in \text{Skew}_n \left( \R \right) $ implies $A \in L \left( SO \left( n \right)  \right) $ or $L \left( O \left( n \right)  \right) $.

Define $\gamma \left( t \right) := \exp t A$ to be a curve of matrices on some manifold. By above, we see that
\begin{align}
    \left( \gamma \left( t \right)  \right)^{T} \left( \gamma \left( t \right)  \right) = I
,\end{align}
and thus $\det \gamma \left( t \right) = 1$ which implies $\gamma \left( t \right) \in SO \left( n \right) $. By construction, $A = \dot{\gamma}\left( t \right) \bigg|_{t=0}$ and thus is tangent to the curve at the identity of $SO \left( n \right) $ suggesting $A \in L \left( SO \left( n \right)  \right) $. Therefore
\begin{align}
   \dim SO \left( n \right) = \dim L \left( SO \left( n \right)  \right) = \dim \left( \text{Skew}_n \left( \R \right)  \right) = \frac{n \left( n- 1\right) }{2}
.\end{align}


\subsection{Group product from Lie bracket}

Recall the Baker-Campbell-Haussdorff (BCH) formula, namely that for $X,Y \in L \left( G \right) $, we have
\begin{align}
    \exp \left( tX \right) \exp \left( tY \right) = \exp \left( tZ \right) 
,\end{align}
where
\begin{align}
    Z = X + Y + \frac{t}{2}\left[ X, Y \right] + \frac{t^2}{12} \left( \left[ X, \left[ X, Y \right]  \right]   + \left[ Y, \left[ X, Y \right]  \right] \right) + \mathcal{O}\left( t^3 \right) 
.\end{align}

One can show this order by order in $t$. As $L \left( G \right) $ is closed under the Lie bracket, $Z \in L \left( G \right) $ and thus $\exp t Z \in G$.

\section{Representation Theory}

Groups and their elements represent transformations under which a system or object is invariant. Representations of groups tell us how the action of the group transforms vectors in a vector space.

We saw $GL \left( n, \mathbb{F} \right) $ as a group of invertible matrices. These matrices are equivalently linear maps (automorphisms) on the vector space $\mathbb{F}^{n}$ with
\begin{align}
    GL \left( n, \mathbb{F} \right) : \mathbb{F}^{n} \to \mathbb{F}^{n}
.\end{align}

We generalize this notation to act on any vector space $V$ such that
\begin{align}
    GL \left( V \right) : V \to V
.\end{align}

If $V$ is finite dimensional, we can choose a basis and recover the original definition.

\subsection{Lie group representations}

\begin{definition}
    A \textbf{representation} $D$ of a group $G$ is a smooth group homomorphism
    \begin{align}
        D : G \to GL \left( V \right) 
    ,\end{align}
    from $G$ to the group of automorphisms on some vector space $V$ called the \textbf{representation space}, associated with $D$.
\end{definition}

That is, $\forall g \in G$, $D \left( g \right) : V \to V$ is an invertible, linear map such that for a vector $v \in V$,
\begin{align}
    v \mapsto D \left( g \right) v
.\end{align}

This map is linear such that
\begin{align}
    D \left( g \right) \left( \alpha v_1 + \beta v_2 \right) = \alpha D \left( g \right) v_1 + \beta D \left( g \right) v_2
,\end{align}
$\forall \alpha, \beta \in \mathbb{F}$, $v_1, v_2 \in V$. Further, the group homomorphism holds such that we have
\begin{align}
    D \left( g_2 g_1 \right) = D \left( g_2 \right) D \left( g_1 \right) 
,\end{align}
$\forall g_1, g_2 \in G$. This group homomorphism property implies that
\begin{align}
    D \left( e \right) = \id_{V}
,\end{align}
and by an identical argument,
\begin{align}
    D \left( g \right)^{-1} = D \left( g^{-1} \right) 
.\end{align}

\begin{definition}
    The \textbf{dimension} of a representation $D$ is the dimension of the representation space $V$ on which it acts.
\end{definition}

If $V$ is finite dimensional, say $\dim V = N$, then $GL \left( V \right) $ is isomorphic to $GL \left( N, \mathbb{F} \right) $.




