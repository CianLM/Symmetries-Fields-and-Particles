\lecture{13}{09/11/2024}{Poincare Representations}

\begin{definition}
    $G$ is the \textbf{semi-direct} product group $H \ltimes N$ provided that the two equivalent statements hold:
    \begin{enumerate}[label=\roman*)]
        \item $N$ and $H$ have trivial intersection, $N \cap H = I$. Then $G = NH$.
        \item Every element of $G$ can be written uniquely as $nh$ for $n \in N$ and $h \in H$.
    \end{enumerate}
\end{definition}

The action of $\left( \tensor{\Lambda}{^{\mu}_\nu}, a^{\mu} \right) $ on $x x\in M_4$ can be written
\begin{align}
    x^{\mu} \mapsto \tensor{\Lambda}{^{\mu}_\nu} x^{\nu} + a^{\mu}
,\end{align}
or equivalently,
\begin{align}
    \mqty( x \\ 1 ) \mapsto \mqty( \Lambda & a \\ 0 & 1 ) \mqty( x \\ 1) = \mqty( \Lambda x + a \\ 1 )
,\end{align}
and then composition becomes
\begin{align}
    \mqty( \Lambda & a \\ 0 & 1 ) \mqty( \Lambda' & a' \\ 0 & 1 ) = \mqty( \Lambda \Lambda' & \Lambda a' + a \\ 0 & 1 )
.\end{align}

Recall $N \triangleleft G \iff g N = Ng$, $\forall g \in G \iff g n g^{-1} \in N$, $\forall n \in N$, $g \in G$. Translations ratify this and thus form a normal subgroup
\begin{align}
    \mqty( \Lambda a \\ 0 & 1 ) \mqty( I & a' \\ 0 & 1 ) \mqty( \Lambda & a \\ 0 & 1 )^{-1} \in T^{1,3}
.\end{align}

But there exists $\Lambda$ and $a$ such that
\begin{align}
    \mqty( \Lambda & a \\ 0 & 1 ) \mqty( \Lambda & 0 \\ 0 & 1 ) \mqty( \Lambda & a \\ 0 & 1)^{-1} \not\in O \left( 1,3 \right) 
.\end{align}

\begin{note}
    Observe that
    \begin{align}
        \mqty( \Lambda & a \\ 0 & 1 ) = \mqty( I & a \\ 0 & 1) \mqty( \Lambda & 0 \\ 0 & 1)
    .\end{align}
\end{note}

Recall that $O \left( 1,3 \right) $ has basis elements given by rotations $J_{i}$ and boosts $K_{i}$ such that
\begin{align}
    M^{0j} = K^{j}, && M^{ij} = \epsilon^{ijk} J^{k}
.\end{align}

The generators for translations are $P^{\sigma}$ which in our $5 \times 5$ notation can be written
\begin{align}
    \widetilde{M}^{\mu \nu} = \mqty( M^{\mu \nu} & 0 \\ 0 & 0 ) && \widetilde{P}^{\sigma} = \mqty( 0 & P^{\sigma} \\ 0 & 0 )
,\end{align}
with $\left( P^{\sigma} \right)^{\beta} = \eta^{\sigma \beta}$. With $6$ generators from $O \left( 1,3 \right) $ and $4$ from $T^{1,3}$ we have a 10 dimensional algebra.

The group elements can then be written using the exponential map with
\begin{align}
    \left( \Lambda, a \right) = \exp \left( a_\sigma P^{\sigma} \right) = \exp \left( a_\sigma P^{\sigma} \right) \exp \left( \frac{1}{2} \omega_{\mu \nu} M^{\mu \nu} \right) 
.\end{align}

We have Lie brackets
\begin{align}
    \left[ M^{\mu \nu}, M^{\rho \sigma} \right]  &= \eta^{\nu \rho} M^{\mu \sigma} - M^{\nu \sigma} + \eta^{\mu \sigma} M^{\nu \rho} - \eta^{\nu \sigma} M^{\mu \rho} \\
    \left[ M^{\mu \nu}, P^{\sigma} \right] &= \eta^{\nu \sigma} P^{\mu} - \eta^{\mu \sigma} P^{\nu} \\
    \left[ P^{\mu}, P^{\nu} \right] = 0
.\end{align}

Later we will treat Casimir elements (those objects which commute with all algebra generators) more properly. Here observe that $P^2 = P_\sigma P^{\sigma}$ commutes with $M^{\mu \nu}$ and $P^{\sigma}$. This is a quadratic Casimir element.

We also have the Pauli-Lubowski pseudovector given by
\begin{align}
    W_{\mu} = \frac{1}{2} \epsilon_{\mu \nu \rho \sigma} M^{\nu \rho} P^{\sigma}
.\end{align}

$W_{\mu} W^{\mu}$ is a quartic Casimir element.

\subsection{Representations of the Poincare group}

There are no finite dimensional representations of the Poincare group which are unitary.

We focus here on representations useful for describing single particle states.

We denote a unitary representation of the Poincare group by by $U$. Then for any $\left( \Lambda, a \right) \in ISO \left( 1,3 \right) $, we have $U \left( \Lambda ,a \right) : V \to V$ with representation space $V$.

As this is a semidirect product, we can write
\begin{align}
    U \left( \Lambda, a \right) = T\left( a \right) U \left( \Lambda \right)
,\end{align}
where $T \left( a \right) = U \left( I,a \right) $ and $U \left( \Lambda \right) = \left( U \Lambda, 0 \right) $.

\begin{note}
    We can write $\left( \Lambda, \Lambda a \right) = \left( \Lambda, 0 \right) ( I, a) = \left( I, \Lambda a \right) \left( \Lambda, 0 \right) $. Thus
    \begin{align}
        U \left( \Lambda \right) T \left( a \right) = T\left( \Lambda a \right) U \left( \Lambda \right) 
    .\end{align}
\end{note}o

Translations are generated by $P^{\sigma}$ and thus
\begin{align}
    \left( 0,a \right) = \exp \left( a_\sigma P^{\sigma} \right)  = e^{a \cdot P}
,\end{align}
where $T \left( a \right) $ is the unitary representation of this group element. Let $\ket{p,s}$ be an eigenvector of $T \left( a \right) $ in a vector space $V$, appropriate for a single particle state. We then have
\begin{align}
    T \left( a \right) \ket{p,s} = e^{i \vb{a} \cdot \vb{p}}
,\end{align}
where $i p^{\sigma}$ is an eigenvalue of $P^{\sigma}$. $p^{\sigma}$ correspond to particle momenta and $s$ is an internal discrete degree of freedom.

Lorentz transforming an eigenvector, we see
\begin{align}
    T\left( a \right) \left( U \left( \Lambda \right) \ket{p,s} \right) &= U \left( \Lambda \right) T \left( \Lambda^{-1} a \right) \ket{p,s} \\
    &= e^{i \left( \Lambda^{-1} a \right) \cdot p} U \left( \Lambda \ket{p,s} \right)  \\
    &= e^{i a \cdot \left( \Lambda p \right) } U \left( \Lambda \right) \ket{p,s}
.\end{align}

This is still an eigenvector with eigenvalue
\begin{align}
    p^{\mu} \to p'^{\mu} = \tensor{\Lambda}{^{\mu}_\nu} p^{\nu}
.\end{align}

Note that $\left( p' \right)^2 = p^2$ as the scalar product is preserved under Lorentz transform. This and the above, make use of $\Lambda^{T} = \Lambda^{-1}$.

For any fixed $p^2$, we have an equivalence class of momentum eigenvectors all related by Lorentz transformation.

We choose some ``standard'' or ``reference'' momentum $k$ such that $k^2 = p^2$. Then $p^{\mu} = \tensor{L \left( p \right)}{^{\mu}_{\nu}}k^{\nu} $ where $L\left( p \right) $ is a Lorentz transformation that takes us to the $k$ frame.

Then we write eigenvectors as
\begin{align}
    \ket{p,s} = U \left( L \left( p \right)  \right) \ket{k,s}
.\end{align}





